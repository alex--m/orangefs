
\documentclass[10pt]{article} % FORMAT CHANGE
\usepackage[dvips]{graphicx}
\usepackage{times}

\graphicspath{{./}{figs/}} 

\pagestyle{plain}

\addtolength{\hoffset}{-2cm}
\addtolength{\textwidth}{4cm}

\addtolength{\voffset}{-1.5cm}
\addtolength{\textheight}{3cm}

\setlength{\parindent}{0pt}
\setlength{\parskip}{11pt}

\title{Trove DBPF Handle Allocator }
\author{PVFS Development Team}

\begin{document}
\maketitle
\begin{verbatim}$Id: handle-allocator.tex,v 1.1 2003-01-24 23:29:18 pcarns Exp $\end{verbatim}

\section{Introduction}

The Trove interface gives out handles -- unique identifiers to trove
objects.  In addition to being unique, handles will not be reused within
a configurable amount of time.  These two constraints make for a handle
allocator that ends up being a bit more complicated than one might
expect.  Add to that the fact that we want to serialize on disk all or
part of the handle allocator's state, and here we are with a document to
explain it all.

\subsection{Data Structures}
\subsubsection{Extents}
We have a large handle space we need to represent efficiently.  This
approach uses extents:
\begin{verbatim}

struct extent {
	int64_t first;
	int64_t last;
};

\end{verbatim}

\subsubsection{Extent List}
We keep the extents (not nescessarily sorted) in the \texttt{extents}
array.  For faster searches, \texttt{index} keeps an index into
\texttt{extents} in an AVL tree. 
In addition
to the extents themselves, some bookkeeping members are added.  The most
important is the \texttt{timestamp} member, used to make sure no handle in
its list gets reused before it should.  \texttt{\_\_size} is only used
internally, keeping track of how big \texttt{extents} is.  

\begin{verbatim}
struct extentlist {
	int64_t __size;
	int64_t num_extents;
	int64_t num_handles;
	struct timeval timestamp;
	struct extent * extents;
};
\end{verbatim}

\subsubsection{Handle Ledger}
We manage several lists.  The \texttt{free\_list} contains all the valid
handles. The \texttt{recently\_freed\_list} contains handles which have been
freed, but possibly before some expire time has passed.  The
\texttt{overflow\_list} holds freed handles while items on the
\texttt{recently\_freed\_list} wait for the expire time to pass.

We save our state by writing out and reading from the three
\texttt{TROVE\_handle} members, making use of the higher level trove
interface. 
\begin{verbatim}
struct handle_ledger {
        struct extentlist free_list;
	struct extentlist recently_freed_list;
	struct extentlist overflow_list;
	FILE *backing_store;
	TROVE_handle free_list_handle;
	TROVE_handle recently_freed_list_handle;
	TROVE_handle overflow_list_handle;
}
\end{verbatim}

\section{Algorithm}
\subsection {Assigning handles}
Start off with a \texttt{free\_list} of one big extent encompassing the
entire handle space.
\begin{itemize}
\item Get the last extent from the \texttt{free\_list} (We hope getting
the last extent improves the effiency of the extent representation)
\item Save \texttt{last} for later return to the caller
\item Decrement \texttt{last}
\item if $ first > last $, mark the extent as empty. 
\end{itemize}

\subsection{returning handles}
\begin {itemize}
\item when the first handle is returned, it gets added to the
    \texttt{recently\_freed} list. Because this is the first item on that
    list, we check the time. 
\item  now we add more handles to the list.  we check the time after $N$            handles are returned and update the timestamp.
\item  Once we have added $H$ handles, we decide the \texttt{recently\_freed}
    list has enough handles.  We then start using the
    \texttt{overflow\_list} to hold returned handles.
\item  as with the \texttt{recently\_freed} list, we record the time that
    this handle was added, updating the timestamp after every $N$
    additions.  We also check how old the \texttt{recently\_freed} list is. 
\item  at some point in time, the whole \texttt{recently\_freed} list is ready
    to be returned to the \texttt{free\_list}.  The \texttt{recently\_freed}
    list is merged into the \texttt{free\_list}, the \texttt{overflow\_list}
    becomes the \texttt{recently\_freed} list and the  \texttt{overflow\_list}
    is empty.
\end{itemize}

\subsection{I don't know what to call this section}

Let $T_{r}$ be the minimum response time for an operation of any sort,
$T_{f}$ be the time a handle must sit before being moved back to the free list, and $N_{tot}$ be the total number of handles available on a server.

The pathological case would be one where a caller
\begin{itemize}
\item fills up the \texttt{recently\_freed} list
\item immediately starts consuming handles as quickly as possible to make for
 the largest possible \texttt{recently\_freed} list in the next pass
\end{itemize}

This results in the largest number of handles being unavailable due to sitting
on the \texttt{overflow\_list}.  Call $N_{purg}$ the number of handles waiting
in ``purgatory'' ( waiting for $T_{f}$ to pass) 
\begin{equation}
N_{purg} = T_{f} / T_{r}
\end{equation}

\begin{equation}
F_{purg} = N_{purg} / N_{tot}
\end{equation}
\begin{equation}
F_{purg} = T_{f} / (T_{r} * N_{tot})
\end{equation}

We should try to collect statistics and see what $T_{r}$ and $N_{purg}$ end up being for real and pathological workloads.


\end{document}
% vim: tw=72
