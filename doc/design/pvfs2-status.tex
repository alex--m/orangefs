%
%
\documentclass[11pt, letterpaper]{article}
\usepackage[dvips]{graphicx}
\usepackage{psfig}
\usepackage{rotating}
\usepackage{times}
\pagestyle{empty}

%
% GET THE MARGINS RIGHT, THE UGLY WAY
%
\topmargin 0.0in
\textwidth 6.5in
\textheight 9.0in
\columnsep 0.25in
\oddsidemargin 0.0in
\evensidemargin 0.0in
\headsep 0.0in
\headheight 0.0in

\title{Current PVFS2 status}
\author{ PVFS2 Development Team }
\date{ Last Updated: August 2003 }

%
% BEGINNING OF DOCUMENT
%
\begin{document}
\maketitle

\tableofcontents

\newpage

\thispagestyle{empty}

\setlength{\parindent}{0.0cm}

\section{Introduction}

This document describes the state of the PVFS2 project.  At this time,
it's far from complete, so some important details are listed below.

% \section{Known limitations}

\subsection{Topology}

Currently, only Homogenous clusters are supported.  Further, multiple
data servers are supported, but not multiple metadata servers.  Thus,
the encouraged topology at this time is a single metadata server with
a small number of data servers.  The node acting as a metadata server
can also act as a data server, but isn't necessarily encouraged if the
resources are available.  Alternatively, for simple testing, a single
machine configured as both metadata and a data server will suffice.
Information on configuring your cluster are provided in the quickstart
document provided by the PVFS2 development team.

\subsection{ROMIO}

The ROMIO interface is not complete.  Both ROMIO and PVFS2 are missing a few
features.  Noncontiguous IO does not work.   Resize does not work.   Retrieving
the file size does not work.  However, contiguous IO, both individual and
collective, and removal of files works,   Collective noncontiguous IO has
passes a few tests, but may or may not work for a given application.

\subsection{pvfstab file}

We currently do not handle bogus entries well.  For now, try not to put file
systems known not to work in the pvfs2tab file.

\subsection{The Linux 2.6.x VFS Interface}

At this time, it's possible to mount a PVFS2 volume through the Linux
VFS interface using 2.6.0-test3 kernels or later.  Common operations
(such as 'ls', 'rm', 'mkdir', 'touch', 'chmod', 'chown', etc) work
similarly to other local filesystems.  However, several operations
(such as 'df') will not work as expected.  Also, while it's possible
to develop and write source code on a mounted PVFS2 volume, it is not
possible to execute any programs on it at this time.

% \section{Experimental features}
% \section{Open issues}
% \section{Good examples}


\end{document}

