%
%
\documentclass[11pt, letterpaper]{article}
\usepackage[dvips]{graphicx}
\usepackage{psfig}
\usepackage{rotating}
\usepackage{times}
\pagestyle{empty}

%
% GET THE MARGINS RIGHT, THE UGLY WAY
%
\topmargin 0.0in
\textwidth 6.5in
\textheight 9.0in
\columnsep 0.25in
\oddsidemargin 0.0in
\evensidemargin 0.0in
\headsep 0.0in
\headheight 0.0in

\title{A Quick Start Guide to PVFS2}
\author{ PVFS2 Development Team }
\date{ Last Updated: August 2003 }

%
% BEGINNING OF DOCUMENT
%
\begin{document}
\maketitle

\tableofcontents

\newpage

\thispagestyle{empty}

\setlength{\parindent}{0.0cm}

\section{How to use this document}
\label{sec:howto}

The quick start guide is intended to be a reference on how to quickly
install and configure a PVFS2 file system.  It is broken down into 
three parts.  The first
describes how to download and compile the PVFS2 software.  The next
section walks through the steps of configuring PVFS2 to store and
access files on a single host, which may be useful for simple testing
and evaluation.  The final section of this document describes how to
install and configure PVFS2 in a true cluster environment with multiple
servers and/or clients.

\subsection{Versions}

This document only applies to alpha snapshots of PVFS2 as of August 2003.

\section{Downloading and compiling PVFS}

\subsection{Dependencies}

The following software packages are currently required by PVFS2:
\begin{itemize}
\item Berkely DB (version 3 or 4)
\item pthreads
\item gcc 2.9.x or newer
\end{itemize}

The following system assumptions are also made:
\begin{itemize}
\item homogeneous Linux environment
\end{itemize}

The optional Linux kernel driver has the following requirements:
\begin{itemize}
\item Linux kernel version 2.6 (test release) or newer
\end{itemize}

ROMIO supports PVFS2.  It is not provided with pvfs2, but can be found as part
of the following MPI implementations:

\begin{itemize}
\item MPICH2-0.94 or newer
\end{itemize}

\subsection{Untarring the packages}

All source code is contained in one tarball: pvfs2-XXX.tgz.  The following example assumes that you will be building in the /usr/src directory, although that is not required:

\begin{verbatim}
[root@testhost /root]# cp pvfs2-XXX.tgz /usr/src
[root@testhost /root]# cd /usr/src
[root@testhost /usr/src]# tar -xzf pvfs2-XXX.tgz
[root@testhost /usr/src]# ln -s pvfs2-XXX pvfs2
[root@testhost /usr/src]# ls -lF
total 476
lrwxrwxrwx   1 root    root         15 Aug 14 17:42 pvfs2 -> pvfs2-XXX/
drwxr-xr-x  12 root    root        512 Aug 14 10:11 pvfs2-XXX/
-rw-r--r--   1 root    root     371535 Aug 14 17:41 pvfs2-XXX.tgz

\end{verbatim}

\subsection{Building and installing the packages}

The default steps for building and installing PVFS2 are simply:

\begin{verbatim}
[root@testhost /usr/src]# cd pvfs2
[root@testhost /usr/src/pvfs2-XXX]# ./configure
[root@testhost /usr/src/pvfs2-XXX]# make
[root@testhost /usr/src/pvfs2-XXX]# make install
\end{verbatim}

Here are some optional configure arguments which may be of interest:
\begin{itemize}
\item --with-prefix=$<$path$>$: installs all files in the specified directory
\item --with-kernel=$<$path to kernel source$>$: this enables compilation of
the PVFS2 Linux kernel driver
\item --with-mpi=$<$path to mpi installation$>$: this enables compilation of MPI
based test programs
\item --with-efence: automatically links in Electric Fence for debugging assistance
\end{itemize}

Also note that the pvfs2 kernel source supports out of tree builds if you 
prefer to use that technique.

\section{Configuring PVFS2 for a single host}
\label{sec:single}

This section documents the steps required to configure PVFS2 on a system
in which a single machine acts as both the client and server for all
PVFS2 operations.  It assumes that you have completed the above sections
on building and installation already.  The hostname of the example machine
is ``testhost'' and will be referenced as such in the following examples.
We will store all PVFS2 data in /tmp/pvfs2-storage-space.  /mnt/pvfs2 will
serve as the mount point for the file system.  For more details about
the purpose of these directories please see the PVFS2 users guide.

\subsection{Server configuration}

Since this is a single host configuration, we only have to configure one
server daemon.  In the original PVFS, the metadata and I/O servers were 
separated into two separate programs (mgr and iod).  PVFS2, however, has 
only a single daemon called pvfs2-server which serves both roles.

The most important part of server configuration is simply generating the 
configuration files.  These can be created using the pvfs2-genconfig 
script.  This is an interactive script which will ask several questions
to determine your desired configuration.  Please pay particular attention
to the listing of the metadata servers and I/O servers.  
In this example we will use ``testhost'' for both.  

The pvfs2-genconfig tool will generate two configuration files.  One is a file system configuration file that will be identical for all servers (if we had more than
one).  The second is a server specific configuration file that will be different for
each server.  We will place both of these files in the /etc directory, so you must run this script as root.  

In this simple configuration, we can accept the default options for every
field.  We will use the hostname ``testhost'' rather than ``localhost'' however.

\begin{verbatim}
bash-2.05b# /usr/bin/pvfs2-genconfig /etc/pvfs2-fs.conf /etc/pvfs2-server.conf
**********************************************************************
        Welcome to the PVFS2 Configuration Generator:

This interactive script will generate configuration files suitable
for use with a new PVFS2 file system.  Please see the PVFS2 quickstart
guide for details.

**********************************************************************

You must first select the network protocol that your file system will use.
The only currently supported options are "tcp" and "gm".

* Enter protocol type [Default is tcp]: 

Choose a TCP/IP port for the servers to listen on.  Note that this
script assumes that all servers will use the same port number.

* Enter port number [Default is 3334]: 

Next you must list the hostnames of the machines that will act as
I/O servers.  Acceptable syntax is "node1, node2, ..." or "node{#-#,#,#}".

* Enter hostnames [Default is localhost]: testhost

Now list the hostnames of the machines that will act as Metadata
servers.  This list may or may not overlap with the I/O server list.

* Enter hostnames [Default is localhost]: testhost

Configured a total of 1 servers:
1 of them are I/O servers.
1 of them are Metadata servers.

* Would you like to verify server list (y/n) [Default is n]? 

Choose a file for each server to write log messages to.

* Enter log file location [Default is /tmp/pvfs2-server.log]: 

Choose a directory for each server to store data in.

* Enter directory name: [Default is /tmp/pvfs2-storage-space]: 

Writing fs config file... Done.
Writing 1 server config file(s)... Done.

Configuration complete!
\end{verbatim}

Now we will create a symbolic link to the correct server configuration file 
(note that you will need to adjust the following command to match the 
hostname of your system):

\begin{verbatim}
bash-2.05b# ls -alh /etc/pvfs2*.conf*
-rw-r--r--  1 root   root      384 Aug 14 17:16 /etc/pvfs2-fs.conf
-rw-r--r--  1 root   root       67 Aug 14 17:16 /etc/pvfs2-server.conf-testhost

bash-2.05b# ln -s /etc/pvfs2-server.conf-testhost /etc/pvfs2-server.conf

bash-2.05b# ls -alh /etc/pvfs2*.conf*
-rw-r--r--  1 root   root      384 Aug 14 17:16 /etc/pvfs2-fs.conf
lrwxr-xr-x  1 root   root       31 Aug 14 17:20 /etc/pvfs2-server.conf -> /etc/pvfs2-server.conf-testhost
-rw-r--r--  1 root   root       67 Aug 14 17:16 /etc/pvfs2-server.conf-testhost

\end{verbatim}

\subsection{Starting the server}

Before you run pvfs2-server for the first time, you must run it with a special 
argument that tells it to create a new storage space if it does not already 
exist.

\begin{verbatim}
bash-2.05b# /usr/sbin/pvfs2-server /etc/pvfs2-fs.conf /etc/pvfs2-server.conf -f
\end{verbatim}

Once the above step is done, you can start the server in normal mode 
as follows:

\begin{verbatim}
bash-2.05b# /usr/sbin/pvfs2-server /etc/pvfs2-fs.conf /etc/pvfs2-server.conf
\end{verbatim}

All log messages will be directed to /tmp/pvfs2-server.log, unless you specified
a different location while running pvfs2-genconfig.  If you would prefer to run 
pvfs2-server in the foreground and direct all messages to stderr, then 
you may run the server as follows:

\begin{verbatim}
bash-2.05b# /usr/sbin/pvfs2-server /etc/pvfs2-fs.conf /etc/pvfs2-server.conf -d
\end{verbatim}

\subsection{Client configuration}

There are two primary methods for accessing a PVFS2 file system.  The
first is the native PVFS2 interface which is made available
through {\tt libpvfs2}.  This also happens to be the same interface used by
ROMIO if you configure your system to use MPI-IO.  The second method
relies on a kernel module to provide standard Linux file system
compatibility.  This interface allows the user to use existing
binaries and system utilities on PVFS2 without recompiling.  We will
cover how to configure both access methods here.

We must create a mount point for the file system as well as
an {\tt /etc/pvfs2tab} entry that will be used by
the PVFS2 libraries to locate the file system.  The {\tt pvfs2tab} file is
analogous to the {\tt /etc/fstab} file that most linux systems use to keep up
with file system mount points.

\begin{verbatim}
[root@testhost /root]# mkdir /mnt/pvfs2
[root@testhost /root]# touch /etc/pvfs2tab
[root@testhost /root]# chmod a+r /etc/pvfs2tab
\end{verbatim}

Now edit this file so that it contains the following, except that you should
substitute your host name in place of ``testhost'':

\begin{verbatim}
pvfs-tcp://testhost:3334/pvfs2-fs /mnt/pvfs2 pvfs 0 0 
\end{verbatim}


\emph{TODO: something about kernel driver}

\subsection{Testing your installation}

PVFS2 currently includes the following tools for manipulating the file system
using the native PVFS2 library: pvfs2-ping, pvfs2-import, pvfs2-ls, and pvfs2-export.  These tools check the health of the file system, import local files, 
list the contents of directories, and export files, respectively.  Their usage 
can best be summarized with the following examples:

\begin{verbatim}
bash-2.05b# ./pvfs2-ping /mnt/pvfs2

(1) Searching for /mnt/pvfs2 in /etc/pvfs2tab...

   Initial server: tcp://testhost:3334
   Storage name: pvfs2-fs
   Local mount point: /mnt/pvfs2

(2) Initializing system interface and retrieving configuration from server...

   meta servers (duplicates are normal):
   tcp://testhost:3334

   data servers (duplicates are normal):
   tcp://testhost:3334

(3) Verifying that all servers are responding...

   meta servers (duplicates are normal):
   tcp://testhost:3334 Ok

   data servers (duplicates are normal):
   tcp://testhost:3334 Ok

(4) Verifying that fsid 9 is acceptable to all servers...

   Ok; all servers understand fs_id 9

(5) Verifying that root handle is owned by one server...

   Root handle: 0x00100000
   Ok; root handle is owned by exactly one server.

=============================================================

The PVFS filesystem at /mnt/pvfs2 appears to be correctly configured.

bash-2.05b# ./pvfs2-ls /mnt/pvfs2/

bash-2.05b# ./pvfs2-import /usr/lib/libc.a /mnt/pvfs2/testfile
PVFS2 Import Statistics:
********************************************************
Destination path (local): /mnt/pvfs2/testfile
Destination path (PVFS2 file system): /testfile
File system name: pvfs2-fs
Initial config server: tcp://localhost:3334
********************************************************
Bytes written: 2555802
Elapsed time: 0.416727 seconds
Bandwidth: 5.848920 MB/second
********************************************************

bash-2.05b# ./pvfs2-ls /mnt/pvfs2/
testfile

bash-2.05b# ./pvfs2-ls -alh /mnt/pvfs2/
drwxrwxrwx    1 pcarns  users            0 2003-08-14 22:45 .
drwxrwxrwx    1 pcarns  users            0 2003-08-14 22:45 .. (faked)
-rw-------    1 root    root            2M 2003-08-14 22:47 testfile

bash-2.05b# ./pvfs2-export /mnt/pvfs2/testfile /tmp/testfile-out
PVFS2 Import Statistics:
********************************************************
Source path (local): /mnt/pvfs2/testfile
Source path (PVFS2 file system): /testfile
File system name: pvfs2-fs
Initial config server: tcp://localhost:3334
********************************************************
Bytes written: 2555802
Elapsed time: 0.443431 seconds
Bandwidth: 5.496690 MB/second
********************************************************

bash-2.05b# diff /tmp/testfile-out /usr/lib/libc.a
\end{verbatim}

\emph{TODO: something about kernel driver}


\section{Installing PVFS2 on a cluster}
\label{sec:cluster}
It is important to have in mind the roles that machines (a.k.a. nodes) will
play in the PVFS system. There are three potential roles that a machine might
play:

    * metadata server
    * I/O server
    * client

A metadata server is a node that keeps up with metadata (such as permissions
and time stamps) for the file system. An I/O server is a node that actually
stores a portion of the PVFS file data. A client is a node that can read and
write PVFS files. Your applications will typically be run on PVFS clients so
that they can access the file system.

A machine can fill one, two, or all of these roles simultaneously. Unlike
PVFS-1, each role requires just the pvfs2-server binary, one cluster-wide
config file, and one node-specific config file.

We currently support just one metadata server (this limit will be raised in the
future).  There can be many I/O servers and clients. In this section we will
discuss the components and configuration files needed to fulfill each role.

We will configure our example system so that the node ``cluster1'' provides
metadata information, eight nodes (named ``cluster1'' through ``cluster8''),
and all nodes can as clients.

\subsection{Server configuration}

Installing PVFS2 on a cluster is quite similar to installing it on a single
machine, so familiarize yourself with Section \ref{sec:single}.  We are going
to generate one master config file and 8 smaller node-specific config files. 

\begin{verbatim}
root@cluster1:~# /usr/local/pvfs2/bin/pvfs2-genconfig  \
	/etc/pvfs2-fs.conf /etc/pvfs2-server.conf
**********************************************************************
        Welcome to the PVFS2 Configuration Generator:

This interactive script will generate configuration files suitable
for use with a new PVFS2 file system.  Please see the PVFS2 quickstart
guide for details.

**********************************************************************

You must first select the network protocol that your file system will use.
The only currently supported options are "tcp" and "gm".

* Enter protocol type [Default is tcp]: tcp

Choose a TCP/IP port for the servers to listen on.  Note that this
script assumes that all servers will use the same port number.

* Enter port number [Default is 3334]: 

Next you must list the hostnames of the machines that will act as
I/O servers.  Acceptable syntax is "node1, node2, ..." or "node{#-#,#,#}".

* Enter hostnames [Default is localhost]: cluster{1-8}

Now list the hostnames of the machines that will act as Metadata
servers.  This list may or may not overlap with the I/O server list.

* Enter hostnames [Default is localhost]: cluster1

Configured a total of 8 servers:
8 of them are I/O servers.
1 of them are Metadata servers.

* Would you like to verify server list (y/n) [Default is n]? y

****** I/O servers:
tcp://cluster1:3334
tcp://cluster2:3334
tcp://cluster3:3334
tcp://cluster4:3334
tcp://cluster5:3334
tcp://cluster6:3334
tcp://cluster7:3334
tcp://cluster8:3334

****** Metadata servers:
tcp://cluster1:3334

* Does this look ok (y/n) [Default is y]? y

Choose a file for each server to write log messages to.

* Enter log file location [Default is /tmp/pvfs2-server.log]: 

Choose a directory for each server to store data in.

* Enter directory name: [Default is /tmp/pvfs2-storage-space]: 

Writing fs config file... Done.
Writing 8 server config file(s)... Done.

Configuration complete!
\end{verbatim}


\end{document}

We have now made all the config files for an 8-node storage cluster:
\begin{verbatim}
root@cluster1:~# ls /etc/pvfs2/foo/
pvfs2-fs.conf               pvfs2-server.conf-cluster5
pvfs2-server.conf-cluster1  pvfs2-server.conf-cluster6
pvfs2-server.conf-cluster2  pvfs2-server.conf-cluster7
pvfs2-server.conf-cluster3  pvfs2-server.conf-cluster8
pvfs2-server.conf-cluster4
\end{verbatim}

Now get those 8 node-specific files (pvfs2-server.conf-clusterN) out to the
nodes.  This is probably the most onerous part of pvfs2 setup, and it's
something we will try to address in future releases, but for now, push them out by hand, renaming them to pvfs2-server.conf.  Here's one idea:

\begin{verbatim}
root@cluster1:~# for i in `seq 1 8`; do
> scp /etc/pvfs2-server.conf-cluster${i] cluster${i}:/etc/pvfs2-server.conf
> done
\end{verbatim}

\subsection{starting the server}
