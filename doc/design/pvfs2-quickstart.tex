%
%
\documentclass[11pt, letterpaper]{article}
\usepackage[dvips]{graphicx}
\usepackage{psfig}
\usepackage{rotating}
\usepackage{times}
\pagestyle{empty}

%
% GET THE MARGINS RIGHT, THE UGLY WAY
%
\topmargin 0.0in
\textwidth 6.5in
\textheight 9.0in
\columnsep 0.25in
\oddsidemargin 0.0in
\evensidemargin 0.0in
\headsep 0.0in
\headheight 0.0in

\title{A Quick Start Guide to PVFS2}
\author{ PVFS2 Development Team }
\date{ Last Updated: September 2003 }

%
% BEGINNING OF DOCUMENT
%
\begin{document}
\maketitle

\tableofcontents

\newpage

\thispagestyle{empty}

\section{How to use this document}
\label{sec:howto}

The quick start guide is intended to be a reference on how to quickly
install and configure a PVFS2 file system.  It is broken down into
three parts.  The first describes how to download and compile the
PVFS2 software.  The next section walks through the steps of
configuring PVFS2 to store and access files on a single host, which
may be useful for simple testing and evaluation.  The final section of
this document describes how to install and configure PVFS2 in a true
cluster environment with multiple servers and/or clients.

\subsection{Versions}

This document only applies to alpha snapshots of PVFS2 as of September
2003.

\section{Downloading and compiling PVFS2}

At this time, tarballs (and CVS access) are provided to trusted
parties.  If you're reading this, you should have a method of
obtaining the PVFS2 source code.  Once the source code is downloaded,
compiling the PVFS2 source code is a matter of running './configure',
followed by 'make' from the top level source directory.  More detailed
instruction for building and installing are provided below.

\subsection{Dependencies}

The following software packages are currently required by PVFS2:
\begin{itemize}
\item Berkely DB (version 3 or 4)
\item aio support (provided by glibc and librt)
\item pthreads
\item gcc 2.9.x or newer
\item GNU Make
\end{itemize}

The following software packages are currently encouraged for use with PVFS2:
\begin{itemize}
\item GNU Libc (glibc) 2.3.2 [ or later ]
\item Linux kernel version 2.6.0-test4 [ or later ]
\end{itemize}

The following system assumptions are also made:
\begin{itemize}
\item A Homogeneous GNU/Linux environment
\end{itemize}

ROMIO supports PVFS2.  It is not provided with pvfs2, but can be found as part
of the following MPI implementations:

\begin{itemize}
\item MPICH2-0.94 or newer
\end{itemize}

\subsection{Untarring the packages}

All source code is contained in one tarball: pvfs2-XXX.tgz.  The
following example assumes that you will be building in the /usr/src
directory, although that is not required:

\begin{verbatim}
[root@testhost /root]# cp pvfs2-XXX.tgz /usr/src
[root@testhost /root]# cd /usr/src
[root@testhost /usr/src]# tar -xzf pvfs2-XXX.tgz
[root@testhost /usr/src]# ln -s pvfs2-XXX pvfs2
[root@testhost /usr/src]# ls -lF
total 476
lrwxrwxrwx   1 root    root         15 Aug 14 17:42 pvfs2 -> pvfs2-XXX/
drwxr-xr-x  12 root    root        512 Aug 14 10:11 pvfs2-XXX/
-rw-r--r--   1 root    root     371535 Aug 14 17:41 pvfs2-XXX.tgz

\end{verbatim}

\subsection{Building and installing the packages}

The default steps for building and installing PVFS2 are as follows:

\begin{verbatim}
[root@testhost /usr/src]# cd pvfs2
[root@testhost /usr/src/pvfs2-XXX]# ./configure
[root@testhost /usr/src/pvfs2-XXX]# make
[root@testhost /usr/src/pvfs2-XXX]# make install
\end{verbatim}

Here are some optional configure arguments which may be of interest:
\begin{itemize}
\item --with-prefix=$<$path$>$: installs all files in the specified directory
(/usr/local/ is the default if --prefix is not specified)
\item --with-kernel=$<$path to kernel source$>$: this enables compilation of
the PVFS2 Linux kernel driver [ Requires Linux Kernel 2.6.0-test4 or later ]
\item --with-mpi=$<$path to mpi installation$>$: this enables compilation of MPI
based test programs
\item --with-efence: automatically links in Electric Fence for debugging assistance
\end{itemize}

Also note that the pvfs2 kernel source supports out of tree builds if you 
prefer to use that technique.

\section{Configuring PVFS2 for a single host}
\label{sec:single}

This section documents the steps required to configure PVFS2 on a system
in which a single machine acts as both the client and server for all
PVFS2 operations.  It assumes that you have completed the above sections
on building and installation already.  The hostname of the example machine
is ``testhost'' and will be referenced as such in the following examples.
IMPORTANT: if you intend to use the provided rc scripts to handle startup
and shutdown of the PVFS2 server, then you must specify a valid hostname 
as reported by the \texttt{hostname} command line tool in the configuration.
For this reason, we recommend that you \emph{not} use ``localhost'' as 
the hostname of your server, even if you intend to only test one machine.
We will store all PVFS2 data in /pvfs2-storage-space.  /mnt/pvfs2 will
serve as the mount point for the file system.  For more details about
the purpose of these directories please see the PVFS2 users guide.

\subsection{Server configuration}

Since this is a single host configuration, we only have to configure one
server daemon.  In the original PVFS, the metadata and I/O servers were 
separated into two separate programs (mgr and iod).  PVFS2, however, has 
only a single daemon called pvfs2-server which serves both roles.

The most important part of server configuration is simply generating the 
configuration files.  These can be created using the pvfs2-genconfig 
script.  This is an interactive script which will ask several questions
to determine your desired configuration.  Please pay particular attention
to the listing of the metadata servers and I/O servers.  
In this example we will use ``testhost'' for both.  

The pvfs2-genconfig tool will generate two configuration files.  One is
a file system configuration file that will be identical for all servers
(if we had more than one).  The second is a server specific configuration
file that will be different for each server.  The server specific files
have the hostname of the server that they belong to appended to the
file name.  This script should be excuted as root, so that we can place
the configuration files in their default /etc/ locations. 

In this simple configuration, we can accept the default options for every
field.  We will use the hostname ``testhost'' rather than ``localhost'' however.

\begin{verbatim}
**********************************************************************
	Welcome to the PVFS2 Configuration Generator:

This interactive script will generate configuration files suitable
for use with a new PVFS2 file system.  Please see the PVFS2 quickstart
guide for details.

**********************************************************************

You must first select the network protocol that your file system will use.
The only currently supported options are "tcp", "gm", and "ib".

* Enter protocol type [Default is tcp]: 

Choose a TCP/IP port for the servers to listen on.  Note that this
script assumes that all servers will use the same port number.

* Enter port number [Default is 3334]: 

Next you must list the hostnames of the machines that will act as
I/O servers.  Acceptable syntax is "node1, node2, ..." or "node{#-#,#,#}".

* Enter hostnames [Default is localhost]: testhost

Now list the hostnames of the machines that will act as Metadata
servers.  This list may or may not overlap with the I/O server list.

* Enter hostnames [Default is localhost]: testhost

Configured a total of 1 servers:
1 of them are I/O servers.
1 of them are Metadata servers.

* Would you like to verify server list (y/n) [Default is n]? 

Choose a file for each server to write log messages to.

* Enter log file location [Default is /tmp/pvfs2-server.log]: 

Choose a directory for each server to store data in.

* Enter directory name: [Default is /pvfs2-storage-space]: 

Writing fs config file... Done.
Writing 1 server config file(s)... Done.

Configuration complete!
\end{verbatim}

\subsection{Starting the server}

Before you run pvfs2-server for the first time, you must run it with a special 
argument that tells it to create a new storage space if it does not already 
exist.  In this example, we must run the server as root in order to create
a storage space in /pvfs2-storage-space as specified in the configuration
files.

\begin{verbatim}
bash-2.05b# /usr/sbin/pvfs2-server /etc/pvfs2-fs.conf \
	/etc/pvfs2-server.conf-testhost -f
\end{verbatim}

Once the above step is done, you can start the server in normal mode 
as follows:

\begin{verbatim}
bash-2.05b# /usr/sbin/pvfs2-server /etc/pvfs2-fs.conf \
		/etc/pvfs2-server.conf-testhost
\end{verbatim}

All log messages will be directed to /tmp/pvfs2-server.log, unless you specified
a different location while running pvfs2-genconfig.  If you would prefer to run 
pvfs2-server in the foreground and direct all messages to stderr, then 
you may run the server as follows:

\begin{verbatim}
bash-2.05b# /usr/sbin/pvfs2-server /etc/pvfs2-fs.conf \
	/etc/pvfs2-server.conf -d
\end{verbatim}

\subsubsection{Automatic server startup and shutdown}
\label{sec:rc}

Like most other system services, PVFS2 may be started up automatically at 
boot up time through the use of rc scripts.  We have provided one such
script that is suitable for use on RedHat (or similar) rc systems.  The 
following example demonstrates how to set this up:

\begin{verbatim}
bash-2.05b# cp /usr/src/pvfs2/examples/pvfs2-server.rc \
    /etc/rc.d/init.d/pvfs2-server
bash-2.05b# chmod a+x /etc/rc.d/init.d/pvfs2-server
bash-2.05b# chkconfig pvfs2-server on
bash-2.05b#  ls -al /etc/rc.d/rc3.d/S35pvfs2-server 
lrwxrwxrwx  1 root  root   22 Sep 21 13:11 /etc/rc.d/rc3.d/S35pvfs2-server \
    -> ../init.d/pvfs2-server
\end{verbatim}

This script will now automatically launch on startup and shutdown to 
ensure that the pvfs2-server is started and stopped gracefully.
To manually start the server, you can run the following command:

\begin{verbatim}
bash-2.05b# /etc/rc.d/init.d/pvfs2-server start
Starting PVFS2 server:                                     [  OK  ]
\end{verbatim}

To manually stop the server:

\begin{verbatim}
bash-2.05b# /etc/rc.d/init.d/pvfs2-server stop
Stopping PVFS2 server:                                     [  OK  ]
\end{verbatim}

\subsection{Client configuration}
\label{subsec:client}

There are two primary methods for accessing a PVFS2 file system.  The
first is the native PVFS2 interface which is made available
through {\tt libpvfs2}.  This also happens to be the same interface used by
ROMIO if you configure your system to use MPI-IO.  The second method
relies on a kernel module to provide standard Linux file system
compatibility.  This interface allows the user to use existing
binaries and system utilities on PVFS2 without recompiling.  We will
cover how to configure both access methods here.

We must create a mount point for the file system as well as
an {\tt /etc/pvfs2tab} entry that will be used by
the PVFS2 libraries to locate the file system.  The {\tt pvfs2tab} file is
analogous to the {\tt /etc/fstab} file that most linux systems use to keep up
with file system mount points.

\begin{verbatim}
[root@testhost /root]# mkdir /mnt/pvfs2
[root@testhost /root]# touch /etc/pvfs2tab
[root@testhost /root]# chmod a+r /etc/pvfs2tab
\end{verbatim}

Now edit this file so that it contains the following, except that you should
substitute your host name in place of ``testhost'':

\begin{verbatim}
tcp://testhost:3334/pvfs2-fs /mnt/pvfs2 pvfs2 default 0 0 
\end{verbatim}

\subsection{Testing your installation}
\label{subsec:testing}
PVFS2 currently includes (among others) the following tools for
manipulating the file system using the native PVFS2 library: pvfs2-ping,
pvfs2-import, pvfs2-ls, and pvfs2-export.  These tools check the health
of the file system, import local files, list the contents of directories,
and export files, respectively.  Their usage can best be summarized with
the following examples:

\begin{verbatim}
bash-2.05b# ./pvfs2-ping /mnt/pvfs2

(1) Searching for /mnt/pvfs2 in /etc/pvfs2tab...

   Initial server: tcp://testhost:3334
   Storage name: pvfs2-fs
   Local mount point: /mnt/pvfs2

(2) Initializing system interface and retrieving configuration from server...

   meta servers (duplicates are normal):
   tcp://testhost:3334

   data servers (duplicates are normal):
   tcp://testhost:3334

(3) Verifying that all servers are responding...

   meta servers (duplicates are normal):
   tcp://testhost:3334 Ok

   data servers (duplicates are normal):
   tcp://testhost:3334 Ok

(4) Verifying that fsid 9 is acceptable to all servers...

   Ok; all servers understand fs_id 9

(5) Verifying that root handle is owned by one server...

   Root handle: 0x00100000
   Ok; root handle is owned by exactly one server.

=============================================================

The PVFS2 filesystem at /mnt/pvfs2 appears to be correctly configured.

bash-2.05b# ./pvfs2-ls /mnt/pvfs2/

bash-2.05b# ./pvfs2-import /usr/lib/libc.a /mnt/pvfs2/testfile
PVFS2 Import Statistics:
********************************************************
Destination path (local): /mnt/pvfs2/testfile
Destination path (PVFS2 file system): /testfile
File system name: pvfs2-fs
Initial config server: tcp://localhost:3334
********************************************************
Bytes written: 2555802
Elapsed time: 0.416727 seconds
Bandwidth: 5.848920 MB/second
********************************************************

bash-2.05b# ./pvfs2-ls /mnt/pvfs2/
testfile

bash-2.05b# ./pvfs2-ls -alh /mnt/pvfs2/
drwxrwxrwx    1 pcarns  users            0 2003-08-14 22:45 .
drwxrwxrwx    1 pcarns  users            0 2003-08-14 22:45 .. (faked)
-rw-------    1 root    root            2M 2003-08-14 22:47 testfile

bash-2.05b# ./pvfs2-export /mnt/pvfs2/testfile /tmp/testfile-out
PVFS2 Import Statistics:
********************************************************
Source path (local): /mnt/pvfs2/testfile
Source path (PVFS2 file system): /testfile
File system name: pvfs2-fs
Initial config server: tcp://localhost:3334
********************************************************
Bytes written: 2555802
Elapsed time: 0.443431 seconds
Bandwidth: 5.496690 MB/second
********************************************************

bash-2.05b# diff /tmp/testfile-out /usr/lib/libc.a
\end{verbatim}

\section{Installing PVFS2 on a cluster}
\label{sec:cluster}
It is important to have in mind the roles that machines (a.k.a. nodes) will
play in the PVFS2 system. There are three potential roles that a machine might
play: metadata server,  I/O server, or client. 

A metadata server is a node that keeps up with metadata (such as permissions
and time stamps) for the file system. An I/O server is a node that actually
stores a portion of the PVFS2 file data. A client is a node that can read and
write PVFS2 files. Your applications will typically be run on PVFS2 clients so
that they can access the file system.

A machine can fill one, two, or all of these roles simultaneously. Unlike
PVFS-1, each role requires just the pvfs2-server binary.  It will consult the
cluster-wide config file and the node-specific config file when it starts up to
know what role pvfs2-server should perform on this machine.

We currently support just one metadata server (this limit will be raised in the
future).  There can be many I/O servers and clients. In this section we will
discuss the components and configuration files needed to fulfill each role.

We will configure our example system so that the node ``cluster1'' provides
metadata information, eight nodes (named ``cluster1'' through ``cluster8'')
provide I/O services, and all nodes act as clients.

\subsection{Server configuration}
We will assume that at this point you have either performed a make install 
on every node, or else have provided the pvfs2 executables, headers, and 
libraries to each machine by some other means.

Installing PVFS2 on a cluster is quite similar to installing it on a single
machine, so familiarize yourself with Section \ref{sec:single}.  We are going
to generate one master config file and 8 smaller node-specific config files. 
Again, remember that it is critical to list correct hostnames for each machine,
and to make sure that these hostnames match the output of the \texttt{hostname} 
command on each machine that will act as a server.

\begin{verbatim}
root@cluster1:~# /usr/local/pvfs2/bin/pvfs2-genconfig  \
	/etc/pvfs2-fs.conf /etc/pvfs2-server.conf
**********************************************************************
        Welcome to the PVFS2 Configuration Generator:

This interactive script will generate configuration files suitable
for use with a new PVFS2 file system.  Please see the PVFS2 quickstart
guide for details.

**********************************************************************

You must first select the network protocol that your file system will use.
The only currently supported options are "tcp" and "gm".

* Enter protocol type [Default is tcp]:

Choose a TCP/IP port for the servers to listen on.  Note that this
script assumes that all servers will use the same port number.

* Enter port number [Default is 3334]: 

Next you must list the hostnames of the machines that will act as
I/O servers.  Acceptable syntax is "node1, node2, ..." or "node{#-#,#,#}".

* Enter hostnames [Default is localhost]: cluster{1-8}

Now list the hostnames of the machines that will act as Metadata
servers.  This list may or may not overlap with the I/O server list.

* Enter hostnames [Default is localhost]: cluster1

Configured a total of 8 servers:
8 of them are I/O servers.
1 of them are Metadata servers.

* Would you like to verify server list (y/n) [Default is n]? y

****** I/O servers:
tcp://cluster1:3334
tcp://cluster2:3334
tcp://cluster3:3334
tcp://cluster4:3334
tcp://cluster5:3334
tcp://cluster6:3334
tcp://cluster7:3334
tcp://cluster8:3334

****** Metadata servers:
tcp://cluster1:3334

* Does this look ok (y/n) [Default is y]? y

Choose a file for each server to write log messages to.

* Enter log file location [Default is /tmp/pvfs2-server.log]: 

Choose a directory for each server to store data in.

* Enter directory name: [Default is /pvfs2-storage-space]: 

Writing fs config file... Done.
Writing 8 server config file(s)... Done.

Configuration complete!
\end{verbatim}

We have now made all the config files for an 8-node storage cluster:
\begin{verbatim}
root@cluster1:~# ls /etc/pvfs2/foo/
pvfs2-fs.conf               pvfs2-server.conf-cluster5
pvfs2-server.conf-cluster1  pvfs2-server.conf-cluster6
pvfs2-server.conf-cluster2  pvfs2-server.conf-cluster7
pvfs2-server.conf-cluster3  pvfs2-server.conf-cluster8
pvfs2-server.conf-cluster4
\end{verbatim}

Now the config files must be copied out to all of the server nodes.  If you 
use the provided (Redhat style) rc scripts, then you can simply copy all
config files to every node; each server will pick the correct config files
based on its own hostname at startup time.  The following example assumes
that you will use scp to copy files to cluster nodes.  Other possibilities
include rcp, bpcp, or simply storing the configuration files on an NFS volume.
Please note, however, that the rc script should be modified if you intend
to store config files in any location other than the default /etc/.

At this time, we also will copy out the example rc script an enable it on
each machine.

\begin{verbatim}
root@cluster1:~# for i in `seq 1 8`; do
> scp /etc/pvfs2-server.conf-cluster${i} cluster${i}:/etc/
> scp /etc/pvfs2-fs.conf cluster${i}:/etc/
> scp /usr/src/pvfs2/examples/pvfs2-server.rc \
    cluster${i}:/etc/rc.d/init.d/pvfs2-server
> ssh cluster${i} /sbin/chkconfig pvfs2-server on
> done
\end{verbatim}

\subsection{starting the servers}

As with the single-machine case, you must run pvfs2-server with a special
argument to create the storage space on all the nodes if it does not already
exist.  Run the following command on every metadata or IO node in the cluster:

\begin{verbatim}
root@cluster1# /usr/sbin/pvfs2-server /etc/pvfs2-fs.conf \
	/etc/pvfs2-server.conf -f
\end{verbatim}

Then once the storage space is created, start the server for real with a
command like this on every metadata or IO node in the cluster:

\begin{verbatim}
root@cluster1# /usr/sbin/pvfs2-server /etc/pvfs2-fs.conf \
	/etc/pvfs2-server.conf
\end{verbatim}

If you want to run the server in the foreground (e.g. for debugging), use the
-d option.

If you wish to automate server startup and shutdown with rc scripts, refer
to the corresponding section \ref{sec:rc} from the single server example.

\subsection{Client configuration}

Setting up a client for multiple servers is the same as setting up a client for a single server.  Refer to section \ref{subsec:client}.  

\subsection {Testing your Installation}

Testing a multiple-server pvfs2 installation is the same as testing a single-server pvfs2 installation.  Refer to section \ref{subsec:testing}

\section{Configuring the PVFS2 Linux VFS Interface}
\label{sec:vfs-configure}

Assuming you've mastered the download and installation steps of
managing the PVFS2 source code, configuring the PVFS2 Linux VFS
Interface is relatively straight forward.  We assume at this point
that you are familiar with running the server and that a PVFS2 storage
space has already been created on the node that you would like to
configure for use with the VFS.  We also assume that you are able to
run a Linux 2.6.0-test4 kernel or later.

The basic steps are as follows:
\begin{itemize}
\item Make sure you are running Linux Kernel version 2.6.0-test4 or
  later
\item Compile the PVFS2 Kernel Module against your running kernel
\end{itemize}

Check to make sure that you are running a Linux 2.6.0-test4 Kernel or
later.  You can do this in the following manner:

\begin{verbatim}
gil:~# cat /proc/version 
Linux version 2.6.0-test4 (root@gil) (gcc version 3.3.1 20030722
(Debian prerelease)) #1 Thu Aug 14 14:32:57 CDT 2003
\end{verbatim}

By issuing that command, we are able to inspect the output to ensure
that we're running an appropriate kernel version.  If your kernel is
older than this version, please download the latest kernel available
at:

\begin{verbatim}
http://www.kernel.org/pub/linux/kernel/v2.6/

(or your favorite mirror site)
\end{verbatim}

Once you're convinced the Linux kernel version is appropriate, it's
time to compile the PVFS2 kernel module.  This kernel module is built
independently from the other parts of PVFS2.  The first step in
building this module is generating the Makefile.  To generate the
Makefile, you need to make sure that you run './configure' with the
'--with-kernel=path' argument.  An example is provided here for your
convenience:

\begin{verbatim}
gil:/usr/src/pvfs2# ./configure --with-kernel=/usr/src/linux-2.6.0-test4
\end{verbatim}

This example assumes that you've already downloaded, compiled, and are
now running the kernel located in the /usr/src/linux-2.6.0-test4
directory.  After this command is issued, build the PVFS2 source tree
if it has not yet been built.

Building the kernel module should now be a trivial task.  From the top
level PVFS2 source directory, you need to first change directory to
the where the code is located, and then run make.  You must perform
this step as root so that the build process can access dependency
information in the kernel source tree.  You should see
output similar to the following if your environment is properly
configured:

\begin{verbatim}
gil:/usr/src/pvfs2# cd src/kernel/linux-2.6/
gil:/usr/src/pvfs2/src/kernel/linux-2.6# make
touch /usr/src/pvfs2/src/kernel/linux-2.6/pvfs2.mod.c
make -C /usr/src/linux SUBDIRS=/usr/src/pvfs2/src/kernel/linux-2.6
modules
make[1]: Entering directory `/usr/src/linux-2.6.0-test4'
make[2]: `arch/i386/kernel/asm-offsets.s' is up to date.
*** Warning: Overriding SUBDIRS on the command line can cause
***          inconsistencies
  CC [M]  /usr/src/pvfs2/src/kernel/linux-2.6/pvfs2.mod.o
  CC [M]  /usr/src/pvfs2/src/kernel/linux-2.6/pvfs2-utils.o
  CC [M]  /usr/src/pvfs2/src/kernel/linux-2.6/waitqueue.o
  CC [M]  /usr/src/pvfs2/src/kernel/linux-2.6/devpvfs2-req.o
  CC [M]  /usr/src/pvfs2/src/kernel/linux-2.6/pvfs2-cache.o
  CC [M]  /usr/src/pvfs2/src/kernel/linux-2.6/dcache.o
  CC [M]  /usr/src/pvfs2/src/kernel/linux-2.6/file.o
  CC [M]  /usr/src/pvfs2/src/kernel/linux-2.6/inode.o
  CC [M]  /usr/src/pvfs2/src/kernel/linux-2.6/dir.o
  CC [M]  /usr/src/pvfs2/src/kernel/linux-2.6/namei.o
  CC [M]  /usr/src/pvfs2/src/kernel/linux-2.6/super.o
  CC [M]  /usr/src/pvfs2/src/kernel/linux-2.6/pvfs2-mod.o
  CC [M]  /usr/src/pvfs2/src/kernel/linux-2.6/pvfs2-bufmap.o
  LD [M]  /usr/src/pvfs2/src/kernel/linux-2.6/pvfs2.o
  Building modules, stage 2.
  MODPOST
  CC      /usr/src/pvfs2/src/kernel/linux-2.6/pvfs2.mod.o
  LD [M]  /usr/src/pvfs2/src/kernel/linux-2.6/pvfs2.ko
make[1]: Leaving directory `/usr/src/linux-2.6.0-test4'
\end{verbatim}

At this point, we have a valid PVFS2 Kernel module.  The module itself
is contained in the output file 'pvfs2.ko' located in the directory
where the previous 'make' command was issued.  You may copy this
module to another location if it better suits your environment.

\section{Testing the PVFS2 Linux VFS Interface}
\label{sec:vfs-test}

Now that you've built a valid PVFS2 kernel module, there are several
steps to perform to properly use the file system.

The basic steps are as follows:
\begin{itemize}
\item Create a mount point on the local filesystem
\item Load the Kernel Module into the running kernel
\item Mount your existing PVFS2 volume on the local filesystem
\item Start the PVFS2 Server application
\item Start the PVFS2 Client application
\item Issue VFS commands
\end{itemize}

First, choose where you'd like to mount your existing PVFS2 volume.
Create this directory on the local file system if it does not already
exist.  Our mount point in this example is /mnt/pvfs2.

\begin{verbatim}
gil:~# mkdir /mnt/pvfs2
\end{verbatim}

Now load the kernel module into your running kernel.  You can do this
by using the 'insmod' program.  You can then verify that the module
was loaded properly using 'lsmod' as follows:

\begin{verbatim}
gil:~# insmod /usr/src/pvfs2/src/kernel/linux-2.6/pvfs2.ko
gil:~# lsmod
Module                  Size  Used by
pvfs2                  52320  0 
\end{verbatim}

Note that the module size observed does NOT need to match that as
shown above.  Also, you can use 'rmmod' to remove the PVFS2 module
after it's been loaded.

Now that the module is loaded, we can mount our PVFS2 file system (and
verify that it's properly mounted) as follows:

\begin{verbatim}
gil:~# mount -t pvfs2 pvfs2 /mnt/pvfs2
gil:~# mount
/dev/hda3 on / type ext3 (rw,errors=remount-ro)
proc on /proc type proc (rw)
devpts on /dev/pts type devpts (rw,gid=5,mode=620)
/dev/hda1 on /boot type ext2 (rw)
pvfs2 on /mnt/pvfs2 type pvfs2 (rw)
\end{verbatim}

NOTE:  You can use 'umount' to unmount the PVFS2 volume.

At this point, we need to start the PVFS2 server and the PVFS2 client
applications before any VFS operations will operate properly on the
mounted PVFS2 volume.  See previous sections on how to properly start
the PVFS2 server if you're unsure.  Starting the PVFS2 client is
covered below.

The PVFS2 client application consists of two programs.
pvfs2-client-core and pvfs2-client.  Do not run pvfs2-client-core by
itself.  pvfs2-client is the PVFS2 client application.  This
application cannot be started unless the PVFS2 server is already
running.  Here is an example of how to start the PVFS2 client:

\begin{verbatim}
gil:/usr/src/pvfs2# cd test/kernel/linux-2.6/
gil:/usr/src/pvfs2/test/kernel/linux-2.6# ./pvfs2-client -f
pvfs2-client starting
Spawning new child process
About to exec ./pvfs2-client-core
Waiting on child with pid 17731
\end{verbatim}

The -f argument is not required.  For reference, this keeps the PVFS2
client application running in the foreground.

Now that the PVFS2 server and PVFS2 client applications are running,
normal VFS operation can be issued on the command line.  An example is
provided below:

\begin{verbatim}
gil:/usr/src/pvfs2/src/kernel/linux-2.6# mkdir /mnt/pvfs2/newdir
gil:/usr/src/pvfs2/src/kernel/linux-2.6# ls -al /mnt/pvfs2/newdir
total 1
drwxr-xr-x    2 root     root            0 Aug 15 13:29 .
drwxr-xr-x    3 root     root            0 Aug 15 13:21 ..
gil:/usr/src/pvfs2/src/kernel/linux-2.6# cp pvfs2.ko
/mnt/pvfs2/newdir/foo
gil:/usr/src/pvfs2/src/kernel/linux-2.6# ls -al /mnt/pvfs2/newdir
total 2
drwxr-xr-x    2 root     root            0 Aug 15 13:29 .
drwxr-xr-x    3 root     root            0 Aug 15 13:21 ..
-rw-r--r--    1 root     root       330526 Aug 15 13:30 foo
\end{verbatim}


\end{document}
