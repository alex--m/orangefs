%
%
\documentclass[11pt, letterpaper]{article}
\usepackage[dvips]{graphicx}
\usepackage{epsfig}
\usepackage{rotating}
\pagestyle{plain}

%
% GET THE MARGINS RIGHT, THE UGLY WAY
%
\topmargin 0.0in
\textwidth 6.5in
\textheight 9.0in
\columnsep 0.25in
\oddsidemargin 0.0in
\evensidemargin 0.0in
\headsep 0.0in
\headheight 0.0in

\title{PVFS\\Configuration of Robust Security}
\author{ Michael Moore & Sam Sampson }
\date{ November 26, 2012 }

%
% BEGINNING OF DOCUMENT
%
\begin{document}
\maketitle

\tableofcontents

\newpage

\thispagestyle{plain}

% \setlength{\parindent}{0.0cm}

\section{Introduction}

\section{Building}

\subsection{configure}
The only change needed when building is to specify the 
\texttt{--enable-security} option. Currently, the default \texttt{configure} 
option is to \emph{disable} the robust security code. The option must be 
specified at configure time to compile support for the robust security code.

\begin{verbatim}
    ./configure --enable-security --prefix=/usr/local/orangefs \
        --with-kernel=/usr/src/kernels/2.6.18-194.17.1.el5-x86_64/
\end{verbatim}

\subsection{build}
Building is performed normally via \texttt{make}.

\begin{verbatim}
    make && make kmod && make install && make kmod_install
\end{verbatim}

\section{Environment}

\subsection{Client}

Each client requires a private key. This allows the servers to verify the 
identity of the client. Each client can use the same private key or a unique 
private key. It is recommended to create a private key for each client to 
mitigate the impact of a compromised private key. To create a private key 
called \texttt{clientkey.pem}:

\begin{verbatim}
    root@pvfs-client ~ # cd /usr/local/orangefs/etc/
    root@pvfs-client ~ # openssl genrsa -out pvfs2-clientkey.pem 1024
    Generating RSA private key, 1024 bit long modulus
    ......................++++++
    .........++++++
    e is 65537 (0x10001)
\end{verbatim}

\subsection{Server}

The server side requires a private key for each server and a key store that 
contains the public keys of each server and client. The private key is 
required to sign the credentials and capabilities a given server creates and 
sends to client. The key store is used to verify the keys used in the 
filesystem. The server private key is created in a similar fashion to the 
client key. Each server can use the same private key or a different private key.
It is recommended to create a private key for each server to mitigate the 
impact of a compromised private key. To create a private key called 
\texttt{serverkey.pem}:

\begin{verbatim}
    root@pvfs-server ~ # cd /usr/local/orangefs/etc/
    root@pvfs-server ~ # openssl genrsa -out pvfs2-serverkey.pem 2048
    Generating RSA private key, 2048 bit long modulus
    ...................................+++
    ............................................+++
    e is 65537 (0x10001)
\end{verbatim}

Second, the key store must be created. Assuming a single server and client, 
as created above, the process to create the key store is to retrieve the public
keys from each generated key and store them in a text file. To store the server
and client public keys in \texttt{keystore}:

\begin{verbatim}
    root@pvfs-server ~ # cd /usr/local/orangefs/etc
    root@pvfs-server ~ # echo "S:pvfs-server" >> keystore
    root@pvfs-server ~ # openssl rsa -in pvfs2-serverkey.pem -pubout >> keystore
    root@pvfs-server ~ # echo "C:pvfs-client" >> keystore
    root@pvfs-server ~ # openssl rsa -in pvfs2-clientkey.pem -pubout >> keystore
\end{verbatim}

The actual hostname of the server and client should be substituted in the echo 
statements above. Note that server hostnames are preceded by "S:", and client 
hostnames with "C:". An example key store file with one server and one client 
would look like:

\begin{verbatim}
    S:pvfs-server
    -----BEGIN PUBLIC KEY-----
    MIIBIjANBgkqhkiG9w0BAQEFAAOCAQ8AMIIBCgKCAQEA2Q8KLwg/CMnBwaJLCuvi
    I7mVhvjupkWZaqwy1nVz+OsWBNZ+eCCwmGjfNyE71qe4SO0kmV1bnhVAzbV28lcT
    15LDQcYxnDJDboN4c1+kO4xcFErO7XcCWmaiUx6GFVI2X82LvhW4zr8ySbFn0LwF
    L1eDE6BKN3Qk5DKZjbZJkcHIs+XpOBnxPWr8e3euece5INoVD1OgDDy0DEICsJHa
    hj2DuvslaOr1qZQeBd63nQshgWvHM77VbsaCjxPOl3gQjRgKbB1jo42cyeVPPjJm
    bhGetGvSibmJr9ca1cBM5MVCYsiDHHvrMEqq3wdYSekVKAIYX3gx1KvDQF2RjhlF
    GwIDAQAB
    -----END PUBLIC KEY-----
    C:pvfs-client
    -----BEGIN PUBLIC KEY-----
    MIGfMA0GCSqGSIb3DQEBAQUAA4GNADCBiQKBgQCw7sUebHfVGgV5M18nXtXqQq5c
    uKZvoaQzNISEYH6tGUYcVLnnu3FLBnxUFnByWZed7FtzR3SFXUlwbRbiQyiiLjMa
    NXlvXE91Cj6UXcDGnvvZE47BEqzYjQ7X8Z+toWMVi37Yj1CcYgEMZwFo7ewpyB7d
    IEHqPO+/oz2wXGK0DwIDAQAB
    -----END PUBLIC KEY-----
\end{verbatim}

\section{Configuration}

\subsection{Client}
The client needs to be able to locate its private key file. The client's 
private key is expected at the default location 
\texttt{SYSCONFDIR/pvfs2-clientkey.pem}. As an example, if you ran 
\texttt{./configure --prefix=/usr/local/orangefs} the default location for the 
client's private key is statically defined as 
\texttt{/usr/local/orangefs/etc/pvfs2-clientkey.pem}. If you are using the kernel 
module and pvfs2-client to access the filesystem you can specify the private 
key path by appending the \texttt{--keypath <key>} argument to pvfs2-client. 
Again, assuming the above client key generation and a prefix of 
\texttt{/usr/local/orangefs}, the pvfs2-client command would be:

\begin{verbatim}
    /usr/local/orangefs/sbin/pvfs2-client \
        --keypath /usr/local/orangefs/etc/clientkey.pem
\end{verbatim}

For \textt{pvfs-*} commands (that use \textt{libpvfs2}) the environment 
variable PVFS2KEY_FILE should be set to specify a non-default path to the 
client private key.

\begin{verbatim}
    export PVFS2KEY_FILE="/usr/local/orangefs/etc/clientkey.pem"
\end{verbatim}

\subsection{Server}

The server requires two new configuration parameters. The path to the private 
key must be specified using the \texttt{ServerKey} parameter. The path to the 
key store must be specified using the \texttt{KeyStore} parameter. The 
parameters can be specified in the \texttt{Defaults} section or the 
\texttt{Server} section of the configuration depending on your needs. Example 
values assuming the server private key and key store generation example above 
( if the files were moved to {\texttt{/usr/local/orangefs/etc/} ):

\begin{verbatim}
    KeyStore /usr/local/orangefs/etc/keystore
    ServerKey /usr/local/orangefs/etc/pvfs2-serverkey.pem
\end{verbatim}

\section{A Word About Permissions}


\end{document}


